%Let's call the packages
\usepackage[utf8]{inputenc}
\usepackage{babel}
\usepackage{dirtytalk}
\usepackage{epigraph}
\usepackage{mathtools}
\usepackage{float}
\usepackage{tablefootnote}
\usepackage{pdfpages}
\usepackage{array}
\usepackage{booktabs}
\usepackage{graphicx}
\usepackage{fontawesome}
\usepackage{caption}
\usepackage{subcaption}
\usepackage[round,authoryear]{natbib}
\usepackage[a4paper,width=150mm, bottom = 25mm, top = 25mm, bindingoffset = 6mm]{geometry}
%Fancy Chapter section. If you want specific configurations for your thesis, change the following lines
%The following configuration is an example of a very nice output that is really fancy and nice
\usepackage{fancyhdr}
\pagestyle{fancy}
\fancyhead{}
\fancyhead[RO,LE]{Title}
\fancyfoot{}
\fancyfoot[LE,RO]{\thepage}
\fancyfoot[LO,CE]{Chapter \thechapter}
\fancyfoot[CO,RE]{Author}
\graphicspath{{images/}}
%Hyper setup
\usepackage[breaklinks, colorlinks, urlcolor=blue, citecolor=blue, linkcolor=blue]{hyperref}
\hypersetup{
    colorlinks = true,
    linkcolor=blue,
    urlcolor=blue,
}
\usepackage{sectsty}
\sectionfont{\sffamily\large}
\subsectionfont{\sffamily\normalsize}

\renewcommand{\baselinestretch}{1.25}
